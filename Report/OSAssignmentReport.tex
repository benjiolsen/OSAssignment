\documentclass{article}
\usepackage{amsmath}
\usepackage{setspace}
\title{COMP 2006 Operating Systems Assignment Semester 1 2019}
\date{06/05/2019}
\author{Benjamin Olsen}

\begin{document}
  \pagenumbering{gobble}
  \maketitle
  \newpage
  \doublespacing
  \tableofcontents
  \singlespacing
  \newpage
  \pagenumbering{arabic}

  \section{Mutual Exclusion}
    \subsection{Mutexes}
        The use of mutex variables, from the POSIX Threads library, allowed me to solve the mutual exclusion problem.
        From the pthread library, creating a variable of type pthread\_mutex\_t, allowed the use of the functions to lock and unlock the mutex.
        This allows for blocking of other threads entering their critical section, as the lock function will only execute if the mutex variable is unlocked.
        Once the thread who has obtained the lock exits their critical section, it calls the unlock function to release the lock for other threads to be able to gain access to their critical section.
        In my program, both the functions/threads for cpu() and task() call to establish locks on the mutex variable "mutex". This ensures that the data being created in the task() function is not edited in the cpu() function if a context switch were to occur at any point.
    \subsection{Condition Varibles}
      In the POSIX Threads library, the "wait" and "signal" functions exist for condition varibles. The wait call allows the thread to lock until the corresponding signal call allows it to exit the lock, and continue with the execution. The use in my program lies in the use of the ready queue. When the ready queue is full, the task function will lock until the cpu function calls for it to complete its execution with the corresponding signal call. From there, the task function will then add two or less tasks into the ready queue, for the cpu tasks to complete. The cpu also calls to the wait function until the task function notifies that it is free to execute on the given tasks in the queue. To ensure that all waiting ends, pthread\_cond\_broadcast() is called, which signals to all waits on all threads, releasing them, rather than the 'at least one' of the signal (from man pages)
    \subsection{Shared Variables}
      To achieve this all, the threads must be able to have shared access to the same variables. The two main ways of doing this are either through the use of global variables, or through the use of variables in a shared heap allocation, which are then passed to each function. For my program, I decided to use the non-global approach, and place all my mutex and condition variables inside of a struct, which is allocated space on the heap, and then passed as a parameter to each function, and therefore thread. I chose this, so as to avoid the use of global variables. I wished to avoid these, as i thought that i could make the program far more modular, and restrict the access to such variables. Whilst this program is never going to be shared or used outside of this assignment, I chose to keep myself to those restrictions to ensure good practice as well. The threads which accesss these, are infact the main thread, and the four created threads. The main thread accesses these variables so as to initialise and prepare them for use, ans then it passes them as an argument when pthread\_create() is called. As this malloced variable is passed to each of the threads, they all access the same heap memory space for this entire struct, which includes the ready queue, the thread\_id's, and the variables to declare the max size of the ready queue, and the flag which is set to exit the loop for the cpu() function.
    \newpage
    \section{Bugs and Issues}
      \subsection{Incompletion}
        At the time of writing, this program is currently incomplete. Whilst the main mutual exclusion problem is mostly solved, there are some remaining issues in the lack of completion of the ready queue, where only tasks up to \#99 are completed. This issue has not been solved, as I am unaware of the cause of the issue, therefore lack the ability to solve for it.\\ As well as this, the scheduler does not save any information to the log file, or calculate any times for any of the tasks. This is incomplete due to the rest of the program taking large amounts of time to complete, and not leaving me much time to finish the other parts.
    \newpage
    \section{Running Program}
      \subsection{Sample Inputs}
        If the program is run in a correct manner, referencing the task file correctly, and inputting a valid ready-queue size as such;
        \begin{equation*}
            ./scheduler\;task\_file\;4
        \end{equation*}
        Then the program will run until completion, and will produce an output in the format;
        \begin{equation*}
            CPU\;\#1\;executed\;task\;\#1\;for:\;17s
        \end{equation*}
        From task one, to the final task, whether it run correctly and reach 100, or whether it fails and ends prematurely at task 99.\\
        \\
        If the scheduler is run in a way which is "incorrect", as per the assignment spec, where it is not in the format;
        \begin{equation*}
            ./scheduler\;task\_file\;(1-10)
        \end{equation*}
        Then, my program informs the user that they have not run the program in the correct manner, and exits.
        \begin{align*}
          &./scheduler\;task\_file\;0 \\
          &Please\;run\;as\;./scheduler\;task\_file\;m \\
          &Where\;m\;is\;the\;maximum\;size\;of\;the\;ready\;queue\;(1-10)
        \end{align*}
\end{document}
